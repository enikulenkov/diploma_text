Кластером называется химическое соединение произвольного количества атомов или молекул.
Наибольший интерес в исследовании кластеров представляет анализ динамического изменения
структуры и свойств кластеров в зависимости от размеров последних.
Многие химические элементы образуют кластеры и присутствуют повсеместно:
медные, серебряные и золотые кластеры в цветном стекле; серебряные кластеры
в фотопленке; молекулярные кластеры в атмосфере и углеродные кластеры в саже.

Так как для больших кластеров (порядка сотен и тысяч атомов) вычисление
динамической структуры с начала образования кластера трудно выполнимо с
использованием современных вычислительных мощностей, то большое внимание
в настоящее время уделяется эмпирическим методам симуляции больших кластеров.
В таких методах важное значение имеет вычисление общей потенциальной энергии
кластера.

Классической задачей, например, является следующая. Для заданного размера кластера
и заданной функции потенциальной энергии необходимо вычислить расположение атомов
(ионов, молекул), соответствующее минимальной потенциальной энергии. Другими словами,
необходимо найти глобальный минимум потенциальной энергии. Так как количество минимумов
растет экспоненциально с увеличением размера кластера, то нахождение глобального минимума
становится достаточно нетривиальной задачей и традиционные методы Монте-Карло и молекулярной
динамики испытывают сложности при его нахождении. Одним из перспективных методов является
применение генетических алгоритмов для решения подобных задач кластерной оптимизации.
