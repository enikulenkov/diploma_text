\newpage

% запись в toc-файл текста и любой latex команды
%\addtocontents{toc}{\vspace{0.5cm}}

\section*{Введение}


% запись в toc-файл текста и номера страницы, где размещен этот текст
\addcontentsline{toc}{section}{\bf Введение}

\underline{\noindent Актуальность темы.} Кластером называется химическое
соединение произвольного количества атомов или молекул.  Наибольший интерес в
исследовании кластеров представляет анализ динамического изменения структуры и
свойств кластеров в зависимости от размеров последних.  Многие химические
элементы образуют кластеры и присутствуют повсеместно: медные, серебряные и
золотые кластеры в цветном стекле; серебряные кластеры в фотопленке;
молекулярные кластеры в атмосфере и углеродные кластеры в саже.

Так как для больших кластеров (порядка сотен и тысяч атомов) вычисление
динамической структуры во времени трудно выполнимо с
использованием современных вычислительных мощностей, то большое внимание
в настоящее время уделяется эмпирическим и полуэмпирическим методам симуляции больших кластеров.

При исследование кластеров часто встречается задача определения их равновесной
конфигурации.  Для заданного размера кластера и заданной функции потенциальной
энергии необходимо вычислить расположение атомов (ионов, молекул),
соответствующее минимальной потенциальной энергии. Другими словами, необходимо
найти глобальный минимум потенциальной энергии. Так как количество минимумов
растет экспоненциально с увеличением размера кластера, то нахождение
глобального минимума становится достаточно нетривиальной задачей и традиционные
методы Монте-Карло и молекулярной динамики испытывают сложности при его
нахождении. Одним из перспективных методов является применение генетических
алгоритмов для решения подобных задач кластерной оптимизации.

Генетический алгоритм - метод поиска, который основан на принципах биологического
естественного отбора. Он использует операторы, аналогичные эволюционным процессам генетического
скрещивания, мутации и селекции для исследования многомерных пространств.

В настоящей работе рассматриваются полуэмпирические математические модели, позволяющие рассчитать
равновесные положения атомов и ионов в кластерах. Учитывая важность построения равновесных пространственных
конфигураций для исследования физических характеристик наносистем, компьютерное моделирование таких конфигураций
представляется нам вполне \underline{актуальным}.

\vspace{+0.3cm}
\underline{\noindent Цели и задачи.} Целью работы являлась апробация и сравнение полуэмпирических подходов для нахождения
равновесных конфигураций молекулярных кластеров. Рассматривались метод молекулярной динамики и
генетический алгоритм. В соответствии с целью работы были поставлены следующие конкретные задачи:
\begin{enumerate}
  \item {\vspace{-0.0cm} Формализация генетического алгоритма и разработка компьютерной программы
      на его основе для расчета равновесной пространственной структуры наносистем.} 
  \item {\vspace{-0.0cm} Создание компьютерной программы для расчета равновесной пространственной структуры наносистем
         с использованием метода молекулярной динамики.} 
  \item {\vspace{-0.0cm} Сравнение результатов работы различных методов по нескольких критериям.TODO: Конкретные критерии}
  \item {\vspace{-0.0cm} Визуализация полученных решений.}
\end{enumerate}

\vspace{+0.0cm}
\underline{Научная новизна:} Компьютерное моделирование равновесной геометрии нанокластеров иридия впервые
выполнено с использованием принципиально различных алгоритмов.


\vspace{+0.0cm} \underline{Основные научные положения, выносимые на защиту:}

\underline{Научная и практическая ценность} данной дипломной работы

\underline{Обоснованность и достоверность} полученных результатов
обеспечивается применением вычислительных методик, основанных на принципиально
различных алгоритмах и соответствующих программных реализациях, что позволяет
повысить достоверность результатов численного моделирования.

%\underline{Структура и объем дипломной работы.} Дипломная работа состоит из введения, трех глав, заключения, двух приложений. Общий объем работы составляет 94 страницы машинописного текста, включая 8 таблиц и 12 рисунков, а также библиографический список использованной литературы из 95 наименований.
\underline{Структура и объем дипломной работы.}
