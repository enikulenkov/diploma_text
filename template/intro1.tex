\clearpage % аналог \newpage

% запись в toc-файл текста и любой латеховской команды
%\addtocontents{toc}{\vspace{0.5cm}}

\section*{Введение1}

% запись в toc-файл текста и номера страницы, где размещен этот текст
\addcontentsline{toc}{abcd}{\bf Введение}
%\addtocontents{toc}{\vspace{0.1cm}}
В последнее время стал заметен интерес к исследованию
поверхностных электромагнитных состояний
фотонно-кристаллических структур. Об этом можно судить
по возросшему количеству публикаций \cite{vinogr, pr_73_113113, pr_76_045305, pr_76_165125, pr_80_043806}.
Однако в этих работах основное внимание уделено теоретическому
рассмотрению проблемы фотонных кристаллов без рассмотрения
конкретных фотонно-кристаллических структур, представляющих интерес для науки и техники.
В настоящей работе исследуется распространение света в одномерной фотонно-кристаллической
структуре $(Si / a - SiO_2)_m$ в направлении нормали к слоям этой структуры,
с учётом поверхностных электромагнитных состояний, в связи с использованием этой системы
в оптическом приборостроении и значительным интересом который проявляют к ней
как экспериментаторы так и теоретики.

Ранее в нашей группе уже были выполнены теоретические исследования
собственных электромагнитных колебаний в одномерной фотонно-кристаллической структуре
$Si / a - SiO_2$\cite{fm_modelir_2009, TimShun_Vestnik_VGTU_2007, TimShunSmir_2008,
TimShunSMirKaz_SPIE_2008, TimShunSMirKaz_SPIE_2009, TimSmirShun_PhysB_2009},
а также были рассчитаны спектры пропускания,
и рассматривалось распространение света в этой структуре методом компьютерного моделирования
Однако в этих работах не принимались во внимание возможные
поверхностные электромагнитные состояния.
В связи с этим, компьютерное моделирование распространения света в $(Si / a - SiO_2)_m$ с учётом
поверхностных электромагнитных состояний представляет собой \underline{актуальную}
задачу.

{\bf Цель работы.}
Целью этого исследования является компьютерное моделирование распространения света в
структуре $(Si / a - SiO_2)_m$ с учётом поверхностных электромагнитных состояний
с использованием конечно разностного метода интегрирования уравнений Максвелла.

{\bf Структура и объём работы.}
Работа состоит из введения 3-х глав заключения
%приложения
и списка литературы.
В первой главе дан краткий обзор общих свойств фотонных кристаллов
В ней рассмотрен механизм образование поверхностных волн на границе
фотонного кристалла.
Во второй главе рассмотрен формализм метода конечных разностей
во временной области и особенностей расчёта с применением
данного метода.
В третьей главе приводятся результаты расчётов пропускания
света с учётом учётом поверхностных электромагнитных состояний.
%В приложении приводятся сведения о единицах и размерностях
%в электромагнитной теории.
