\clearpage
\addcontentsline{toc}{abcd}{\bf Список литературы}

\begin{thebibliography}{99}

\bibitem{Knight}
Knight J.C., Birks T A, Russell P. St. J., Atkin D.M.  Opt. Lett 21 1547 (1996).

\bibitem{Russell}
Russell P. St. J. Science 299 358 (2003).

\bibitem{Bowden}
Bowden C. M., Zheltikov A. M. (Eds.) Nonlinear Optics of Photonic.
Crystals Feature issue of J. Opt. Soc. Am. B 19 no. 9 (2002).

\bibitem{Ivanov}
Иванов А. А., Алфимов М. В., Желтиков А. М. Успехи физических наук.
174 743 (2004).

\bibitem{Zheltikov}
Желтиков А. М. УФН 174 73 (2004).

\bibitem{Reeves}
Reeves W.H., Skryabin D. V., Biancalana F., Knight J. C., Russell P. St. J.,
Omenetto F. G., Efimov A., Taylor A. // J Nature 424 511 (2003).

\bibitem{Fedotov}
Fedotov A.B., Zheltikov A.M., Tarasevitch A.P., von der Linde D. // Appl. Phys.
B 73 181 (2001).

\bibitem{Ivcenko}
Е.Л. Ивченко, А.Н. Поддубный., Резонансные трехмерные фотонные кристаллы,
Физика твердого тела, 2006, том 48, вып. 3

% Кособукин
\bibitem{Kosob} Кособукин В.А. Фотонные кристаллы. //Окно в МикроМир. -- 2002. --№4

\bibitem{Yablonovitch}
Yablonovitch E., Inhibited spontaneous emission in solid-state
physics and electronics. // Physical Review Letters. - 1987, - vol.58, no.20, -  pp.2059-62.

\bibitem{Kushwaha}
Band-gap engineering in two-dimensional periodic photonic crystals. Manvir S. Kushwaha

\bibitem{SmirnovJ}
Смирнов Ю.В., Тимощенко Ю.К. Расчеты собственных мод электромагнитного поля в волноведущих системах на основе двумерных фотонных кристаллов // Материалы IV международного семинара "Физико-математическое моделировани систем". Воронеж: Изд-во ВГТУ, 2007. С.131-134.

\bibitem{dos_nano}
Плотность фотонных состояний в оптических наноматериалах и управление энергетическими уровнями атомов/Р.Х. Гайнутдинов, М.А. Хамадеев, Е. В. Зайцева, М.Х.Салахов//Наносистемы: физика, химия, математика, 2012, 3 (1), С. 56–63

\bibitem{Kosevich}
Основы механики кристаллической решетки. А.М. Косевич. М. Наука. 1972

\bibitem{ElTheorMet}
Достижения электронной теории металлов / Под ред. П. Цише, Г. Леманна. - М.: Мир. - 1984. - Т. 2. – 652 с.

\bibitem{Nemoskalenko}
Немошкаленко В. В. Методы вычислительной физи-ки в теории твердого тела. Электронные состояния в неидеальных кристаллах / В. В. Немошкаленко, Ю. Н. Кучеренко. - Киев: Наукова Думка, 1986. - 296 с.

\bibitem{Patrini}
Patrini M. Optical response of one-dimensional $(Si/SiO_2)_m$ photonic crystals / M. Patrini, M. Gali., M. Belotti // Journal of Applied Physics. - 2002. - Vol. 92, No 4, - P. 1816 -1820.

\bibitem{TimoshenkoLightProp}
Timoshenko Yu. K. Light propagation in one-dimensional photonic finite systems $(Si/a-SiO_2)_m$ with defects / Timoshenko Yu. K., Shunina V. A., Yu. V. Smirnov, O. V. Ka-zarina // Nanophotonic Materials V (Conference of SPIE Sympo-sium on NanoScience + Engineering, 10-14 August 2008, San Diego, USA), Proc. of SPIE. – 2008. - Vol. 7030. – P. 703018-1 - 703018-6.

%\bibitem{Sakoda}
%Sakoda K. Optical Properties of Photonic Crystals / K. Sakoda. - Vol. 80 of Springer Series in Optical Sciences, Spring-er, Berlin, Germany, 2nd edition, 2005. - 300 p.

\bibitem{Methfessel}
Methfessel M. High-precision sampling for Brillouin-zone integration in metals / M. Methfessel, A. T. Paxton // Phys. Rev. B, 1989-II. -- Vol. 40, No. 6. -- P. 3616--3621.

\bibitem{Marzari}
Marzari N. Thermal Contraction and Disordering of the Al(110) Surface / N. Marzari, D. Vanderbild, A. De Vita, M. C. Pane // Phys. Rev. Lett. -- 1999. -- Vol. 82, No. 16. -- P. 3296--3299.

\bibitem{TimSmirSav1D}
Тимошенко Ю.К., Смирнов Ю.В., Савченко С.Ю. Компьютерное моделирование плотностей электромагнитных состояний
в одномерных фотонных кристаллах на основе кремния // Физико-математическое моделирование систем: материалы
VIII Междунар. семинара. Воронеж: ВГТУ, 2012. Ч. 4. С. 83-86.

\bibitem{Mallawany}
El-Mallawany R.A.H. Tellurite glasses handbook (New York: CRC Press, 2002).

\bibitem{Sokolov}
Соколов В.О. Численное моделирование фотонно-кристаллических световодов из теллуритно-вольфраматного стекла для применения в параметрических волоконных устройствах. /
В.О. Соколов, В.Г. Плотниченко, В.О. Назарьянц, Е.М. Дианов // Квантовая Электроника, 2006, Том 36, № 1, с. 67-72.




\bibitem{Lehmann}
On the Numerical Calculation of the Density of States and Related Properties. G. Lehmann, M. Taut. physica status solidi (b)
Volume 54, Issue 2, pages 469–477, 1 December 1972

\bibitem{Phys_en} Физическая энциклопедия, т.2. М. Советская
энциклопедия. 1990

\bibitem{Phys_values} Физические величины. Справочник. Под ред.
И.С. Григорьева, Е.З. Мейлихова М., Энергоатомиздат. 1991

\bibitem{Davidov} Теория твердого тела. А.С.Давыдов. М. Наука. 1976

\bibitem{gavr}
Гавриленко, В.И. Оптические свойства полупроводников. Справочник. Киев. Наукова думка, 1987, 607 c.







%\bibitem{vinogr}
%А. П. Виноградов, А. В. Дорофеенко, А. М. Мерзликин, А. А. Лисянский,
%Поверхностные состояния в фотонных кристаллах, УФН, 180:3 (2010), 249–261
%
%%\bibitem{bas} Bassani F., Knox R.S., Fowler W.B.,
%%Band structure and electronic properties of AgCl and AgBr,
%%Phys.~Rev., {\bf A137}, N 4, A1217, 1965.
%\bibitem{pr_73_113113}
%N. Malkova., C. Z. Ning.,
%Shockley and Tamm surface states in photonic crystals,
%Phys.~RevB., {\bf B} 73, 113113, 2006
%
%\bibitem{pr_76_045305}
%N. Malkova., C. Z. Ning.,
%Interplay between Tamm-like and Shockley-like surface states in photonic crystals,
%Phys.~RevB., {\bf B} 76, 045305, 2007
%
%\bibitem{pr_76_165125}
%Jaroslaw Klos.,
%Conditions of Tamm and Shockley state existence in chains of resonant cavities in a
%photonic crystal,
%Phys.~RevB., {\bf B} 76, 165125, 2007
%
%\bibitem{pr_80_043806}
%N. Malkova., I. Hromada., X. Wang., G. Bryant., Z. Chen.,
%Transition between Tamm-like and Shockley-like surface states
%in optically induced photonic superlattices,
%Phys.~RevB., {\bf B} 80, 043806, 2009
%
%\bibitem{fm_modelir_2009}
%Расчет некоторых оптических характеристик фотонно-кристаллических структур на основе кремния
%/А.А.Должников, Р.Ю.Таболин, Ю.В.Смирнов,В.А.Шунина, Ю.К.Тимошенко//
%Материалы VI Международного семинара ``Физико-математическое моделирование систем''. --
%Воронеж (27--28 ноября 2009) -- С. 27--34.
%
%% Ий
%\bibitem{Yee} К. S. Yee. Numerical solution of initial boundary value problems involving
% Maxwell's equations in isotropic media. IEEE Trans. Antennas Propagat. AP-14. 302. 1966
%
%
%
%\bibitem{TimShun_Vestnik_VGTU_2007} Тимошенко Ю.К. Спектры пропускания и компьютерное
%моделирование распространения света в одномерной дефектной
%фотонно~-~кристаллической структуре Si/a-SiO2 / Ю.К.Тимошенко, В.А. Шунина,
%Е.В. Панитков // Вестник ВГТУ. --  2007. --  Т. 3, № 8. --  С. 93--94.
%
%\bibitem{TimShunSmir_2008} Тимошенко Ю.К. Исследование влияния дисперсии
%диэлектрической проницаемости на спектры пропускания одномерных дефектных
%фотонно~-~кристаллических сред на основе кремния / Ю.К. Тимошенко, В.А. Шунина,
%Ю.В.Смирнов // Изв. РАН. Сер. Физика. -– 2008. –- Т. 72, № 9. -– С. 1308--1310.
%
%\bibitem{TimShunSMirKaz_SPIE_2008} Timoshenko Yu. K. Light propagation in
%one~-~dimensional photonic finite systems $(Si/a-SiO_2)_m$ with defects /
%Timoshenko Yu. K., Shunina V. A., Yu. V. Smirnov, O. V. Kazarina
%// Nanophotonic Materials V (Conference of SPIE Symposium on
%NanoScience + Engineering, 10--14 August 2008, San Diego, USA), Proc. of SPIE.
%– 2008. - Vol. 7030. – P. 703018-1 -- 703018-6.
%
%\bibitem{TimShunSMirKaz_SPIE_2009} Timoshenko Yu.K. Calculations of transmission
%spectra in 1D photonic structures accounting polariton effects / Timoshenko
%Yu.K., Shunina V.A., Yu. V. Smirnov, O. V. Kazarina // Nanophotonic Materials
%VI (Conference of SPIE Symposium on NanoScience + Engineering, 2--6 August
%2009, San Diego, USA), Proc. of SPIE. -– 2009. -- Vol. 7393. -– P. 73930V-1 --
%73930V-6.
%
%\bibitem{TimSmirShun_PhysB_2009} Timoshenko Yu. K. Computer simulation of some optical
%properties of one~-~dimensional photonic finite systems $(Si/a-SiO_2)_m$ with
%defects / Yu. K.Timoshenko, Yu. V. Smirnov, V. A. Shunina // Physica B:
%Condensed Matter. – 2009. -– Vol. 404, No. 23--24. -– P. 5207--5208.
%
%%\bibitem{scop} Scop P.M.,
%%Band structure of silver chloride and silver bromide,
%%Phys.~Rev., {\bf A139}, N 3, 934, 1965.
%
%%\bibitem{nun} G.S. Nunes, P.B. Allen, J.S. Martins,
%%Pressure--induced phase transitions in silver halides,
%%Phys.~Rev., {\bf B57}, N 9, 5098, 1998.
%
%%\bibitem{vic} R.H. Victora,
%%Calculated electronic structure of silver halide crystals,
%%Phys.~Rev., {\bf B56}, 4417, 1997
%.
%%\bibitem{kop} Ю.В. Копаев, С.Н. Молотков, С.С. Назин,
%%Размерный эффект в квантовых проводах кремния.
%%Письма в ЖЭТФ, {\bf 55}, вып.12, 696, 1992.
%
%%\bibitem{oka} Y. Okamoto,
%%Nachr. Acad. Wiss. G\"ottingen, Math.--Phys., Kl. IIa,
%%Math.--Phys.--Chem., Abt., 275, N 14, 1956.
%
%%%%%%%%%%%%%%%%%%%%%%%%%%%%%%%%%%%%%%%%%%%%%%%%%%%%%%%%%%%%%%%%%%%%%%%%%%%
%%
%% Примеры ссылок на литературу
%% (современный стандарт для диссертаций - 2004 г.)
%%
%
%% Статья
%%\bibitem{Wil} Wilson W. Doctrinal rationality after Woollin
%%/ W. Wilson // Mod. Law rev. - L., 1999. - Vol. 62, ?3. - P. 448-463.
%
%% Статья в сборнике трудов
%%\bibitem{Shav} Шавкун И.Г. К истории семантико-синтаксической сочетаемости
%%глагола-сказуемого и в страдательном залоге / И. Г. Шавкун // Сб.
%%науч. тр. МГПИИЯ. - М., 1990. - Вып. 194. - С. 230 - 248.
%
%% Тезисы доклада
%%\bibitem{Vov} Вовна В. И. Сравнительный анализ электронного строения и
%%химической связи молекул фторидов d - и p - элементов по
%%результатам фотоэлектронной спектроскопии и квантовохимических
%%расчетов // VIII Всесоюзный симпозиум по химии неорганических
%%фторидов: Тез. докл. - М.: Наука, 1987. - С. 18.
%
%% Книга одного автора
%%\bibitem{Art} Артемьева Т.В. История метафизики в России XVIII века /
%%Т. В. Артемьева. - СПб.: Алетейя, 1996. - 317 с.
%
%% Книга трех авторов
%%\bibitem{Bob} Бобров А. А. Твой учебник здоровья: Физкульт.-спортив. тесты
%%(ФСТ) Главы Ступин. р-на Моск. обл.: Учеб. пособие / А.А. Бобров,
%%О. М. Боброва, Л. И. Еременская. - М.: Совет. спорт, 2001. - 101
%%с.
%
%% Книга, имеющая более трех авторов
%%\bibitem{Sb} Сборник задач по математике для поступающих в вузы / В.К. Егерев и
%%др.; Под ред. М.И. Сканави.- Минск: Высш. шк., 1990. - 526 с.
%
%% Отдельный том
%%\bibitem{Sav} Савельев И.В. Курс общей физики. Т. 1. Механика.
%%Молекулярная физика: Учеб. пособие для студентов вузов / И. В.
%%Савельев. - 2-е изд., перераб. - М.: Наука, 1982. - 432 с.
%
%%Составная часть книги
%%\bibitem{Mak} Макаров И.М. Робототехника и научно-технический прогресс /
%%И. М. Макаров // Робот. Компьютер. Гибкое производство. - М.:
%%Наука, 1990. - С.5-16.
%
%
%% Диссертация
%%\bibitem{Petr} Петренко Т.Ф. Оптическое поглощение $AgBr$ : Дис. ...
%%канд. физ.- мат. наук / Т. Ф. Петренко. - М., 1990. - 145 с.
%
%
%%\bibitem{Inet} Griffith A.L. Computer simulations [Electronic resource]
%% / A. L. Griffith . // Education Policy Analysis Archives.- 1995.- vol.3,
%% no. 1.-Доступно из URL: http://olam.ed.asu.edu/ [Дата обращения: 12 февр].
%
%
%
%
%\bibitem{Taflove}
%Taflove A., Hagness S. H. Computational Electrodynamics: The Finite
%Difference Time-Domain Method.-- Boston: Artech House, 2005.
%
%\bibitem{Berenger}
%Berenger J. P. A perfectly matched layer for the absorption of electro-
%magnetic waves // J. Comput. Phys.-- 1994.-- Vol. 114.-- Pp. 185–200.
%
%\bibitem{Taflove2}
%Taflove A., Review of FD-TD Numerical Modeling of Electromagnetic Wave Scattering and Radar Cross Section. Proceedinggs of the IEEE, vol. 77, no 5, May 1989
%
%\bibitem{Luebbers}
%Raymond Luebbers, David Steich, and Karl Kunz. FDTD Calculation of  Scattering from Frequency-Dependent Materials. IEEE transactions on antennas and propagation, vol. 41, no. 9, september 1993
%
%\bibitem{Zheng}
%H.-X. Zheng, X.-Q. Sheng, E. K.-N. Yung. Computation of scattering from anisotropically coated bodies using conformal FDTD. Progress In Electromagnetics Research, PIER 35, 287–297, 2002
%
%\bibitem{Johnson}
%Steven G. Johnson, J. N. Winn, R. D. Meade, and J. D. Joannopoulos.
%A Brief Survey of Computational Photonics. October 24, 2007
%********************************************************************************************




\end{thebibliography}
