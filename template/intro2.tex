\clearpage % аналог \newpage
\section*{Введение}
% запись в toc-файл текста и номера страницы, где размещен этот текст
\addcontentsline{toc}{abcd}{\bf Введение}
В настоящее время ещё не установилась окончательная терминология относящаяся к световодам
с пространственно-периодическим показателем преломления. Используются такие термины, как
брегговские, фотонно-кристаллические (ФК) и т.п. световоды. Для определённости мы будем называть
фотонно-кристаллическими световодами структуры, представляющие собой фрагмент
двумерного фотонного кристалла с пространственно--периодической зависимостью показателя преломления
в поперечном сечении и дефектом в центральной области световода.

Фотонно-кристаллические волокна обладают рядом уникальных свойств, открывающих новые
возможности для передачи электромагнитного излучения на большие расстояния \cite{Knight, Russell},
а также для нелинейно-оптического преобразования лазерных импульсов \cite{Bowden, Ivanov, Zheltikov}.
Такие волокна позволяют существенно расширить по сравнению с обычными волокнами частотный диапазон,
соответствующий одномодовому режиму волноводного распространения излучения \cite{Bowden}.
Уникальность ФК-волокон для лазерной физики, нелинейной оптики и оптических технологий обусловлена
возможностью управления дисперсией волноводных мод за счёт изменения их структуры \cite{Reeves}
и высокой степенью локализации электромагнитного излучения в сердцевине микроструктурированных
волокон \cite{Fedotov}, связанной со значительной разностью показателя преломления сердцевины
и эффективным показателем преломления микроструктурированной оболочки. Управление
дисперсионными свойствами волноводных мод открывает новые возможности в области оптических
телекоммуникаций и оптике сверхкоротких импульсов. Высокая степень локализации излучения в
сердцевине волокна приводит к радикальному увеличению эффективности нелинейно-оптических
взаимодействий и позволяет наблюдать новые нелинейно-оптические явления.

В последнее время активно исследуются и изготавливаются фотонно-кристаллические световоды на основе кварцевого стекла $v-SiO_2$.
C другой стороны, известны материалы с более высокими значениями показателя преломления,
в которых, кроме того, минимальным оптическим потерям соответствует область более длинных волн чем
в $v-SiO_2$ К таким материалам относятся стекла на основе $TeO_2$.

%К таким материалам в частности, относятся стекла на основе $TeO_2$ и кристаллические галогениды серебра
%($n\approx2.0...2.1$) и стекла на основе халькогенидов мышьяка $As_2S_3$ и $As_2Se_3$ ($n\approx2.4...2.9$)
%используемые для изготовления обычных волоконных световодов.

Ранее уже проводилось изучение таких систем в научном центр волоконной оптики при Институте общей физики
им. А.М.Прохорова Российской академии наук (НЦВО при ИОФ РАН). Работы велись под руководством академика Дианова.
Оптические свойства таких систем теоретически изучались, опираясь только лишь на расчёт собственных колебательных
мод в приближении расширенной элементарной ячейки. Причём расчёты производились только для центра зоны Бриллюэна.
В связи с этим были поставлены задачи:
\begin{enumerate}
  \item Разработка формализма, алгоритмов и программ для расчёта фотонных зон и плотностей ЭМ состояний
  для одномерных и двумерных фотонных кристаллов.
  \item Оценка корректности методов расчёта плотностей ЭМ состояний для одномерных кристаллов.
  \item Расчёт фотонной зонной структуры и плотностей ЭМ состояний одномерных ФК на основе кремния и двумерных фотонных кристаллов на основе Si, $Si/a-SiO_2$ и теллуритно-вольфраматного стекла.
  \item Анализ полученной расчётной информации с целью предсказания положения полос пропускания на частотной шкале для исследованных ФК.
\end{enumerate}
