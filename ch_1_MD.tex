\subsection{Формализм метода молекулярной динамики}
\label{sec:1b}

Метод молекулярной динамики~--- метод, в котором временная эволюция системы
взаимодействующих атомов или частиц отслеживается интегрированием их уравнений
движения.

Концептуально классическая молекулярная динамика включает:

1. решение уравнений движения Ньютона для взаимодействующих атомов или молекул;

2. статистический анализ полученных траекторий в фазовом пространстве системы.

Пусть некоторая система состоит из n атомов и известна их потенциальная энергия
взаимодействия $U(r_1, r_2, \cdots , r_n).$
Согласно уравнению движения классической механики:
\begin{equation}
\label{one}
m_{i}a_{i}=f_i=-\frac{\partial}{\partial r_i}U,
\end{equation}
для каждого атома $i$ можно найти его траекторию $r_i(t)$. Здесь $m_i$ - масса
атома, $a_i=\partial^2r_i/\partial t^2$ - ускорение,
$f_i$ - сила, действующая на атом $i$ со стороны остальных атомов. При
известных начальных условиях для координат $r_i(0)=r_i^0$ и
скоростей $\upsilon_i(0)=\upsilon_i^0$  задача является
детерминистической, то есть временная эволюция координат $r_i(t)$
однозначно определена. Таким образом, в компьютере можно проследить "жизненный
путь" всех атомов в системе. Например, непосредственно увидеть процесс
встраивания в пленку атома, осаждаемого из газа; развитие каскада атомных
соударений при внешнем облучении; распространение трещины при разрушении;
образование зародышей новой фазы; детали атомных перестроек в жидкости и другие
явления, которые иным способом исследовать либо трудно, либо невозможно.

При изучении макроскопических свойств системы сами по себе траектории атомов
$r_i(t)$ не представляют особого интереса. Они используются для
нахождения плотности вероятности заполнения фазового пространства системы.
Траектории атомов дают набор конфигураций, распределенных в соответствии с
некоторым статистическим ансамблем. В этом смысле метод молекулярной динамики является
методом статистической механики, которая прокладывает мостик между микроскопическим
поведением и термодинамикой.

Применимость уравнений движения классической механики (\ref{one}) для атомарных
систем имеет известные ограничения. Длина волны де Бройля $\lambda$ должна быть
меньше среднего межатомного расстояния а :
\begin{equation}
\label{two}
\lambda \ll a, \lambda=\sqrt{2\pi\hbar/m\kappa_B T},
\end{equation}
здесь $T$ - температура, $\kappa_B$ - постоянная Больцмана. Соотношение
(\ref{two}) нарушается для легких атомов, таких как $H$, $He$. Кроме того,
квантовые эффекты становятся существенными при низких температурах для любых
атомарных систем, что следует учитывать при интерпретации результатов, полученных
методом молекулярной динамики, в этой температурной области.

В методе молекулярной динамики силы находятся из потенциальной энергии системы
U (\ref{one}). Можно сказать, что правильный выбор потенциальной энергии
межатомного взаимодействия является ключевым элементом всего метода
молекулярной динамики. Как уже было сказано в параграфе \ref{sec:1a}, в данной
работе был использован потенциал Саттона--Чена.

Метод молекулярной динамики позволяет найти глобальный минимум потенциальной энергии
многоатомной системы. В то время как различные подходы минимизации потенциальной энергии
системы по координатам ядер этого не гарантируют.
