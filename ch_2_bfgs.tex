\subsection{Алгоритм Бройдена-Флетчера-Гольдфарба-Шанно}
\label{sec:2d}

На каждой итерации генетического алгоритма для каждого полученного кластера из
популяции ищется локальный минимум функции потенциальной энергии. Такой подход
впервые был предложен в работе Дивена и Хо \cite{Deaven1995} и позволяет значительно
уменьшить пространство возможных решений для поиска генетическим алгоритмом.
В данной работе для локальной минимизации кластеров на каждой итерации был использован
алгоритм Бройдена-Флетчера-Гольдфарба-Шанно.

Алгоритм Бройдена-Флетчера-Гольдфарба-Шанно (BFGS) - один из наиболее широко применяемых
квазиньютоновских методов. В квазиньютоновских методах не вычисляется гессиан функции.
Вместо этого гессиан оценивается приближенно, исходя из сделанных до этого шагов.

Описание данного алгоритма и особенности его реализации приведены в \cite{Fletcher1987}.
Существует также реализация алгоритма BFGS в рамках библиотеки GSL (GNU Scientific library)
\cite{gsl}, опирающаяся на данные, приведенные в \cite{Fletcher1987}, которая
и была использована в данной работе.

Пусть решается задача оптимизации функционала $f(x)$. Методы второго порядка решают данную задачу
итерационно, с помощью разложения функции в полином второй степени:

\begin{equation}
  f(x_{k}+p) = f(x_k) + \nabla f^{T}(x_k)p + 1/2p^{T}H(x_k)p,
\end{equation}
где $H$~-- гессиан функционала $f$ в точке $x$.

Зачастую вычисление гессиана трудоемки, поэтому BFGS алгоритм вместо настоящего значения $H(x)$ вычисляет
приближенное значение $B_k$, после чего находит минимум полученной квадратичной задачи:

\begin{equation}
  p_k = -B_{k}^{-1} \nabla f(x_k).
\end{equation}

Затем осуществляется поиск вдоль данного направления точки, для который выполняются условия Вольфе-Пауэлла:

\begin{equation}
  \label{search_constr}
  \begin{array}{l}
    f(x_k + \alpha_{k}p_k) \leq f(x_k) + c_1 \alpha_k \nabla f_{k}^{T}p_k, \\
    |\nabla f(x_k + \alpha_{k}p_k)^{T}p_k| \leq c_2 |\nabla f_{k}^{T}p_k|
  \end{array}
\end{equation}

Поиск представляет собой итеративный процесс, который генерирует последовательность оценок
$\lbrace \alpha_{k}^{i} \rbrace$. В данной реализации $\alpha_{k}^{0} = 0$, а $\alpha_{k}^{1}$
является параметром алгоритма и задает начальный шаг поиска.

В качестве начального приближения гессиана можно брать любую невырожденную, хорошо обусловленную матрицу.
В данной реализации была использована единичная матрица в качестве начального приближения.
Приближенное значение гессиана на следующем шаге вычисляется по формуле:

\begin{equation}
  B_{k+1} = B_k - \frac{B_k s_k s_{k}^{T} B_k}{s_{k}^{T} B_k s_k} + \frac{y_k y_k^{T}}{y_k^{T} s_k},
\end{equation}
где $s_k = x_{k+1} - x_k$~-- шаг алгоритма на итерации,
$y_k = \nabla f_{k+1} - \nabla f_{k}$~-- изменение градиента на итерации.

При инициализации работы алгоритма библиотека GSL принимает на вход несколько параметров:

\begin{itemize}
  \item{fdf - структура, содержащая информацию об оптимизируемой функции: размерность, указатель на неё,
    а также указатель на функцию, считающую её производную. В данном случае оптимизируемой функцией
    является общая потенциальная энергия кластера.}
  \item{x   - первоначальное приближение точки минимума. В данном случае это текущая пространственная
    конфигурация кластера}
  \item{step\_size - начальный шаг при поиске вдоль заданного направления точки, удовлетворяющей условиям
    (\ref{search_constr}) (значение $\alpha_{k}^{1}$)}
  \item{tol - значение константы $c_2$ в (\ref{search_constr}})
\end{itemize}

Параметры {\it step\_size} и {\it tol} сильно влияют на точность получаемых результатов и время выполнения
алгоритма BFGS, поэтому они являются и параметрами, задающимися пользователем при инициализации
генетического алгоритма.

Алгоритм BFGS требует вычисления градиента функции во время своей работы, поэтому для функции полной
потенциальной энергии были произведены расчеты градиента, которые приведены в приложении %TODO: приложение 1. 
