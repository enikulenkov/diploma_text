%
% ПРЕАМБУЛА КУРСОВОЙ РАБОТЫ
%

\documentclass [12pt,a4paper]{article}

\usepackage[T2A]{fontenc}
%\usepackage[cp1251]{inputenc}       % кодировка (обязательно!)
\usepackage[utf8]{inputenc}
\usepackage[russian]{babel} % Система переносов (обязательно!)
\usepackage[backend=biber,
  language=auto,
  babel=other,
  bibstyle=gost-numeric,
]
{biblatex}

\usepackage{amsmath,amssymb}
%\usepackage{mathtext}    % если нужны русские буквы в формулах
%\usepackage{mathrsfs}    % рукописные буквы в формулах
\usepackage{graphicx}    % пакет для включения рисунков
\usepackage{geometry}
\usepackage{indentfirst} % отступ в первом абзаце раздела
%\usepackage{cite}        % cite.sty (ссылки на литературу в виде [1-5,9])
\usepackage[centerlast,large]{caption2}
\usepackage{multirow}
\usepackage{fancyhdr}
\usepackage{setspace}
\usepackage{url}
\usepackage[usenames,dvipsnames]{color}

% замена двоеточия на точку в номерах рисунков и таблиц
\renewcommand{\captionlabeldelim}{.}

% в каждую строку таблицы вставляется \strut
\renewcommand{\arraystretch}{1.1}

%установка полей
\geometry{verbose,a4paper,tmargin=2cm,bmargin=2cm,lmargin=3cm,rmargin=1.5cm}
\headheight=5mm         % место для колонтитула
\headsep=10mm           % отступ после колонтитула
%\mathsurround=2pt      % доп. пробел слева и справа от формулы в тексте

\def\baselinestretch{1.5} % увеличение межстрочного интервала в 1.5 раза
%\renewcommand{\baselinestretch}{1.5}
\frenchspacing  % одинаковые пробелы между словами
\parindent=3em  % абзацный отступ
\tolerance=400  % мера разреженности строки

\makeatletter

%  для размещения точки после номера в заголовках разделов
\renewcommand{\@seccntformat}[1]{\csname the#1\endcsname.\hspace{1em}}

% ??? (нужно для \addcontentsline{toc}{abcd}{\bf Введение})
\newcommand{\l@abcd}[2]{\hbox to\textwidth{#1\hfill #2}}
\newcommand{\inlinecode}[1]{{\it #1}}

\renewcommand{\subsubsection}{\@startsection{subsubsection}{3}
{\parindent}{3.5ex plus 1ex minus .2ex} {2.3ex plus
.2ex}{\normalfont\large\bfseries}}

\makeatother


% Put page number at the header by center
\pagestyle{fancy}

\fancyhf{}
\chead{\thepage}

\renewcommand{\headrulewidth}{0pt}
\renewcommand{\footrulewidth}{0pt}

\fancypagestyle{plain}{
        \fancyhf{}
        \chead{\thepage}}
