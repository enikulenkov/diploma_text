\subsection{Потенциал Саттона--Чена}
\label{sec:1a}

При проведении экспериментов над пространственными структурами металлических кластеров
довольно часто используют потенциальную функцию Саттона--Чена (TODO: link to the article).
Полная потенциальная энергия при таком подходе рассчитывается следующим образом:
\begin{equation}
\label{Etot}
E_{tot}=\sum_i E_{i} = \sum_i \epsilon \left[\frac12\sum_{i\ne j} V(r_{ij})-c \sqrt{\rho_i}\right],
\end{equation}

Где
\begin{equation}
\label{V(r)}
V(r_{ij})=(a/r_{ij})^n,
\end{equation}

\begin{equation}
\label{p_i}
\rho_i=\sum_{i\ne j}\left(\frac{a}{r_{ij}}\right)^m
\end{equation}

Здесь $r_{ij}$  - раccтояние между атомами $i$ и $j$, $с$ - положительный
безразмерный параметр, $\epsilon$ - параметр, имеющий размерность энергии, $a$
- параметр длины, и $m$ и $n$ - положительные целые числа. По причинам, описанным
ниже, $n$ должно быть больше чем $m$. Парный потенциал, $V$, является просто
отталкивающим, и термин N-частичный просто связен. Обозначим значение $\rho^s$,
заданного суммой в равенстве (\ref{V(r)}) для атома, принадлежащего свободной
поверхности. Связный вклад, который этот атом делает к полной энергии
поверхности, таков $\epsilon=-\epsilon c \sqrt{\rho s}$. Если
дополнительный атом поместить выше поверхности и в разделение $R$ от нашего
поверхностного атома, значение $\rho^s$ изменяется на:
$$-\epsilon c[\rho^s+(a/R)^m]^{1/2}$$

Для больших значений $R$, по сравнению с $a$, мы можем расширить квадратный корень:
\begin{equation}
\label{R4}
[\rho^s+(a/R)^m]^{1/2}\backsimeq \sqrt{\rho s}+ \frac{(a/R)^m}{2\sqrt{\rho s}}
\end{equation}

Из выражения (\ref{R4}) видно, что когда атом приближается к поверхности, он
взаимодействует в больших разделениях попарным способом, хотя величина этого
парного потенциала зависит от числа соседей каждого поверхностного атома. Как
разделение, $R$, уменьшается, расширение в выражении (\ref{R4}) становится все
более и более недействительным, и взаимодействие гладко изменяется на
N-частичную форму. Таким образом, выбирая $m$=6 в выражении (\ref{p_i}), мы можем
достигнуть нашей цели взаимодействий дальнего действия между двумя группами
атомов, описываемых парным потенциалом $1/r^6$, тогда как в маленьких
разделениях N-частичное взаимодействие в природе. Кроме того, переход между
этими двумя пределами гладкий и непрерывный.

% TODO: Table with parameters of different metals parameters? 
